\documentclass[10pt]{beamer}
\usetheme{Bergen}
\usepackage{caption}
\usepackage{graphicx}
\usepackage{amsmath}
\usepackage{amssymb}

\title{Van der Pol Oscillator}
\subtitle{AE 425 Project}
\institute{Indian Institute of Technology Bombay}
\date{\today}
\author{Abhilash Kulkarni}

\begin{document}

	\begin{frame}[plain]
		\maketitle
	\end{frame}

	\begin{frame}
		\frametitle{Table of Contents}
		\tableofcontents 
	\end{frame}

	\section{Introduction}
		\begin{frame}
			\frametitle{Introduction}
			The Van der Pol oscillator is a non-conservative oscillator with non-linear damping. The oscillator is defined by the following second-order differential equation:

			\begin{equation}
			\frac{d^2x}{dt^2} - \mu(1 - x^2)\frac{dx}{dt} + x = 0
			\end{equation}

			where x is the position coordinate which is a function of the time t, and μ is a scalar parameter indicating the nonlinearity and the strength of the damping.
		\end{frame}
	\section{Two-dimensional form}
		\begin{frame}
		\frametitle{Two-dimensional form}
		\textbf{Liénard's theorem} is used to prove that the van der Pol Oscillator has a limit cycle. Applying the Liénard transformation: $y = x - x^3/3 - \frac{\dot{x}}{\mu}$, the Van der Pol oscillator can be written in its two-dimensional form:
		\begin{equation}
		\dot{x} = \mu(x - \frac{1}{3}x^3 - y)
		\end{equation}
		\begin{equation}
		\dot{y} = \frac{1}{\mu}x
		\end{equation}
		Another form which is used is based on the transformation $y = \dot{x}$:
		\begin{equation}
		\dot{x} = y
		\end{equation}
		\begin{equation}
		\dot{y} = \mu(1 - x^2)y - x
		\end{equation}
		\end{frame}
\pagebreak
\begin{frame}
	
\end{frame}

\bibliography{bib_file}{}
\bibliographystyle{plain}

\end{document}
